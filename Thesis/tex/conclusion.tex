\section{Conclusion}
In this paper we presented a method to delineate linear vegetation elements from LiDAR point clouds. Growing regions based on the rectangularity proved a useful way to segment different objects in the vegetation which could be checked for linearity. Accuracies were high, with overall accuracies of 0.87 and 0.90, in a research area with linear and non-linear vegetation patches of different shapes and sizes. The main limitations are the balance between computation time and precision, which can be solved by using high performance computing paradigms, and the difficulty of setting good rules for merging regions. Despite these limitations the method was largely successful in its goal and is ready to be upscaled to larger areas using high performance computing paradigms like cloud computing. The ecological value of this research can then be further explored. Furthermore the method could also be used for different elongated objects, like roads, ditches and waterways.